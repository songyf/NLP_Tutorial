%%%%%%%%%%%%%%%%%%%%% chapter.tex %%%%%%%%%%%%%%%%%%%%%%%%%%%%%%%%%
%
% sample chapter
%
% Use this file as a template for your own input.
%
%%%%%%%%%%%%%%%%%%%%%%%% Springer-Verlag %%%%%%%%%%%%%%%%%%%%%%%%%%
%\motto{Use the template \emph{chapter.tex} to style the various elements of your chapter content.}
\chapter{对话理解与智能质检}
\label{basic} % Always give a unique label
% use \chaptermark{}
% to alter or adjust the chapter heading in the running head
本节首先给出对话理解任务的定义,然后介绍对话理解的主要方法。接下来以智能质检为例,讲述对话理解是怎么落地和实现的。
\section{对话理解}
\subsection{什么是对话理解}
对话理解是指希望计算机跟人一样,具备自然语言理解的能力,从对话内容中挖掘对话意图,理解对话意图,用户情绪识别等。例如,在客服与用户交互的对话中,用户询问“今天的天气如何”,这里就是一个“询问天气”的意图。
这里对话可以包含语音对话和文本对话,如果是语音对话,我们一般可以利用语音自动识别技术将语音转为文本。后续我们要讨论的内容是文本的对话理解。
\subsection{技术路线分类}
一般而言,文本的对话理解从技术角度上可以分为两类:文本匹配和文本分类。

\textbf{文本匹配}~~~文本匹配的目标是得到$f(text_1, text_2)$的语义匹配得分,其中$text_1$和$text_2$是输入的文本,$f$是文本匹配模型。

\textbf{文本分类}~~~文本分类的目标则是得到${g(text)}$的类别标签,其中$text$是输入的文本,$g$是文本分类模型。在任务型对话中,类别标签就是意图。

\section{应用案例:智能质检}
对话理解在智能客服,智能质检有着广泛的应用。下面以智能质检为例,阐述对话理解相关技术是怎么应用的。
\subsection{什么是智能质检}

智能质检使用人工智能算法,分析坐席呼叫场景下人工客服与客户的对话,实现全量质量检查,提高人工客服的服务质量和客户的满意度。
智能质检系统的输入是一通人工客服和客户对话的录音,输出是质检报表,显示该录音在不同质检项的合格情况。质检项的重要性通过质检项的分数来决定。
智能质检无需人工介入,节省质检人力,覆盖率高(100\%),提升质检效率,降低漏检错检率。
\begin{figure}[h]
\centering
\includegraphics[scale=0.4]{./img/chapter4_b/qic_workflow.jpg}
\caption{智能质检基本流程图}
\label{fig1}
\end{figure}

从智能质检的基本流程图中我们可以发现,对话录音经过语音识别模块之后,我们得到了客户和人工客服之间的对话文本。质检员配置了质检项之后,我们将对话文本输入质检模型,最后得到了质检报表。
\subsection{实现方案与应用状况}
\textbf{实现方案}~~~假设我们定义了质检项要求客服在对话中“核实用户的工作地址”,比如“你的公司地址在哪里”。在冷启动时,一种简单的实现方案是通过规则配置“算子+逻辑操作符”或者正则表达式,如果录音文本中满足匹配条件,则命中该质检项。

如果我们有标注数据,就可以使用文本匹配和文本分类的方法。计算$f$(录音文本,质检例句)的语义匹配得分或者$g$(录音文本)的质检项标签。我们通过数据驱动的方法让模型越来越聪明,业务方只需要提供标注数据就能进行质检,不需要人工定义规则,模型具有一定的泛化能力。但遇到bad case没有基于规则的方法容易修复,另外需要标注数据积累到一定规模才能发挥模型的优势。

\textbf{应用状况}~~~当前智能质检的应用可以包含离线质检和坐席实时质检。离线质检是指结合语音识别和自然语言处理技术,对海量录音数据进行批量的智能化分析。离线质检在质检过程无需人工介入,可以提供内容质检,敏感词识别等质检结果。坐席实时质检是指在人工客服和客户通话过程中,提供实时质检功能,辅助人工客服判断客户情绪和实时分析对话过程的信息,及时提醒人工客服从而使客户获得更好的服务。现在智能质检的产品形态包含SaaS云服务和私有化部署。SaaS级产品部署,让中小企业也能够享受智能质检带来的高效与便捷,克服了采购费用高部署周期长的问题。

未来随着多方业务的使用,可以基于联邦学习进行智能质检,在满足数据安全和私隐保护的前提下,通过模型的参数梯度共享,获得了把所有数据放在一起训练的效果,使得不同的业务方合作共赢,建立更准确的数据模型。

