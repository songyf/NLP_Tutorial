%%%%%%%%%%%%%%%%%%%%% chapter.tex %%%%%%%%%%%%%%%%%%%%%%%%%%%%%%%%%
%
% sample chapter
%
% Use this file as a template for your own input.
%
%%%%%%%%%%%%%%%%%%%%%%%% Springer-Verlag %%%%%%%%%%%%%%%%%%%%%%%%%%
%\motto{Use the template \emph{chapter.tex} to style the various elements of your chapter content.}
\chapter{NLP发展现状与应用领域}
\label{basic} % Always give a unique label
% use \chaptermark{}
% to alter or adjust the chapter heading in the running head

\section{定义简介}
自然语言处理(Natural Language Processing,简称NLP),属于计算机科学与语言学的交叉学科,所以又称计算语言学;它是用计算机来理解、处理、运用人类语言的学科。人类通过自然语言进行沟通协作,可以说如果没有语言人类的智能将无从谈起,它是人类区别于动物的重要标志。也可以说,只有当计算机具备了准确的自然语言的理解处理能力时,才算真正实现了人工智能。
\begin{itemize}
\item 研究内容:NLP研究内容主要包括词法分析、句法分析、语义分析、篇章理解、机器翻译等。
\item 应用场景:NLP广泛应用信息系统方方面面。例如:手写体识别、光学符号识别、语音识别、语音合成、信息检索、机器翻译、对话系统等。
\item 关联学科:NLP紧密相关的研究领域包括机器学习、数据挖掘、知识图谱等;紧密相关的学科包括信息论、语言学、计算机科学等。
\end{itemize}

NLP研究范围涉及自然语言的形态学、语法学、语义学和语用学等几个层次。
\begin{itemize}
\item 形态学(morphology):研究词的内部结构,包括屈折变化和构词法两个部分。
\item 语法学(syntax):研究句子结构成分之间的相互关系、句子序列的组成规则。
\item 语义学(semantics):研究各级语言单位(词素、词组、句子、段落、片等)的意义,以及与语音、语法、语境的关系等等,其重点在探明符号与其所指对象之间的关系。
\item 语用学(pragmatics):研究在不同上下文下的语句应用,以及上下文对语句理解所产生的影响。大概来说,语用学研究范围问题是很广,重点在于研究包括直指、会话隐含、预设、语言行为、话语结构等。
\end{itemize}

NLP面临的两大难题是歧义消解、未知语言现象。
\begin{itemize}
\item 歧义消解:在自然语言的词法、句法、语义等各个层次中存在大量的歧义现象。比如“什么是一个词”,这就是NLP面临的一个难题。因为不同的人对词语粒度、标准有不同的理解。再比如,语言中存在大量一词多义的现象,在上下文语境中如何准确找到对应的词义,这些都是NLP研究面临的实实在在的难题。
\item 未知语言现象:未知语言现象主要由两个方面的原因导致。第一点,人类语言一直处于不断演化过程中,同一个语言表达,在新的时空环境下,可能已经不再是以前的含义;而且,由于信息网络的发达,人们构造、传播新语言现象的能力大大增强,比如互联网上每天都在涌现新的语言词汇。第二点,在NLP研究中,由于整理收录能力、知识表达能力等现实因素的制约,实际中并没有一种可以准确、全面表达人类语言知识工程或工具。
\end{itemize}

\section{发展历史}
NLP发展历史中存在两种不同的的研究方法:基于规则的理性主义;基于统计的经验主义。它们对语言的不同理解,体现了它们不同的哲学思想。
\begin{itemize}
\item 理性主义:认为自然语言是由语言规则来产生和描述的;因此他们相信,只要能够用适当的形式将人类语言规则表示出来,就能够理解人类语言。
\item 经验主义:认为语言知识可以从语言数据中获取,只要建立有效的统计模型就可以理解语言;因此他们相信,如果有足够多语言数据用于统计,就能够理解人类语言。
\end{itemize}

NLP发展历史可以总结为5个时期:
\begin{enumerate}
\item 经验主义萌芽时期:时间大约到20世纪50年代。这个时期NLP或多或少具有经验主义色彩。例如,1913年马尔科夫提出马尔科夫模型\cite{markov1913example}的时候,就曾经计算过长诗中元音与辅音出现的频度概率;再比如,1948年,香农把离散马尔科夫的概率模型应用于语言自动机\cite{shannon1948mathematical}的时候,也曾统计过英语字母的频率。

\item 经验主义低谷时期:时间从1956年~20世纪90年代。1956年,乔姆斯基首先提出使用有限状态机来刻画自然语言\cite{chomsky1956three}。具体来说,就是使用数学的代数、集合论为基础核心,将各种语言现象统一抽象为代数、集合上的运算规则。形式语言理论影响深远,在此后很长一段时期,很多学者逐步完善并扩展了形式语言理论。这段时期,NLP领域的研究方法几乎完全被理性主义主导,经验主义被打入谷底。

\item 经验主义复苏时期:时间从20世纪50年代末到90年代初期。虽然这期间,例行理性主义占据主流,但是有学者已经开始思考引入基于语料库的统计方法到NLP研究中;这其中的代表是1967年诞生的联机语料库Brown Corpus\cite{kuvcera1967computational}。另外,和NLP紧密相关的机器学习方法得到了较大的发展;比如这段时期,诞生了贝叶斯模型、最大熵\cite{jaynes1957information}、维特比算法\cite{viterbi1967error}、隐马尔可夫模型\cite{stratonovich1965conditional}等等,NLP研究有了更多理论工具可供使用。

\item 经验主义爆发时期:时间20世纪90年代中期~2010年左右。时间处于20世纪90年代前期,此刻经验主义已经处于全面复苏的前夜。一方面,由于机器学习领域诞生了很多新理论和方法,推动了NLP快速发展;另一方面,计算机的存储容量、计算能力已经极大提升,使得很多计算量偏大的机器学习方法逐渐实用。受益于这两方面,经验主义终于开始全面复苏,迎来了一个前所未有的黄金时期。20世纪90年代以后,以语料库和统计学习为基础,基于机器学习的词法分析、句法分析、机器翻译、语音识别等研究不断涌现。

\item 经验主义现代时期:时间从2003年左右至今。这段时期,经验主义再上高峰,具有更加鲜明的特色,神经网络、深度学习的是目前NLP研究的关键词。Ruder在博文\cite{ruder2018review}在中总结了这段时期里NLP的研究趋势,其中以神经语言模型(Nerual Language Model,NLM)、多任务学习(Multi-task Learning)、词嵌入(Word Embedding)、Seq2Seq(Sequence To Sequence Model)模型、注意力(Attention)机制、基于记忆的网络(Memory-based Networks)、预训练语言模型(Pretrained Language Models)等这个时期重要的里程碑。

\end{enumerate}

\section{应用领域}
自然语言处理研究的内容非常广泛,应用范围也非常广泛,如下举例一些常见的应用场景:
\begin{itemize}
\item 机器翻译:利用计算机实现自然语言(英语、汉语等)之间的自动翻译。
\item 自动摘要:利用计算机自动地从原始文档中提取全面准确地反映该文档中心内容的简单连贯的短文。
\item 文本分类:在预定义分类体系下,根据文本的特征,将给定文本于一个或多个类别相关联的过程。
\item 情感分类:根据文本所表达的含义和情感将文本划分为褒扬或者贬义的两种或几种类型,是对作者倾向性、观点、态度的划分,因此也称倾向性分析。
\item 信息抽取:从非结构化或半结构的自然语言文本中提取出于某个主题相关的实体、关系。事件等事实信息,并且形成结构化信息输出。
\item 信息检索:用户输入一个表述需求信息的查询字段,系统回复一个包含所需要信息的文档列表。其核心技术在于索引构建和相关性计算。
\item 问答系统:接受用户自然语言形式描述的问题,从大量异构数据中查找或者推断用户问题答案的信息检索系统。
\end{itemize}


