%%%%%%%%%%%%%%%%%%%%% chapter.tex %%%%%%%%%%%%%%%%%%%%%%%%%%%%%%%%%
%
% sample chapter
%
% Use this file as a template for your own input.
%
%%%%%%%%%%%%%%%%%%%%%%%% Springer-Verlag %%%%%%%%%%%%%%%%%%%%%%%%%%
%\motto{Use the template \emph{chapter.tex} to style the various elements of your chapter content.}
\chapter{NLP发展现状与应用领域}
\label{basic} % Always give a unique label
% use \chaptermark{}
% to alter or adjust the chapter heading in the running head

\section{定义简介}
自然语言处理(Natural Language Processing,简称NLP),属于计算机科学与语言学的交叉学科,所以又称计算语言学;它是用计算机来理解、处理、运用人类语言的学科。人类通过自然语言进行沟通协作,可以说如果没有语言人类的智能将无从谈起,它是人类区别于动物的重要标志。也可以说,只有当计算机具备了准确的自然语言的理解处理能力时,才算真正实现了人工智能。
\begin{itemize}
\item 研究内容:NLP研究内容主要包括词法分析、句法分析、语义分析、篇章理解、机器翻译等。
\item 应用场景:NLP广泛应用信息系统方方面面。例如:手写体识别、光学符号识别、语音识别、语音合成、信息检索、机器翻译、对话系统等。
\item 关联学科:NLP紧密相关的研究领域包括机器学习、数据挖掘、知识图谱等;紧密相关的学科包括信息论、语言学、计算机科学等。
\end{itemize}

NLP研究范围涉及自然语言的形态学、语法学、语义学和语用学等几个层次。
\begin{itemize}
\item 形态学(morphology):研究词的内部结构,包括屈折变化和构词法两个部分。
\item 语法学(syntax):研究句子结构成分之间的相互关系、句子序列的组成规则。
\item 语义学(semantics):研究各级语言单位(词素、词组、句子、段落、片等)的意义,以及与语音、语法、语境的关系等等,其重点在探明符号与其所指对象之间的关系。
\item 语用学(pragmatics):研究在不同上下文下的语句应用,以及上下文对语句理解所产生的影响。大概来说,语用学研究范围问题是很广,重点在于研究包括直指、会话隐含、预设、语言行为、话语结构等。
\end{itemize}

NLP面临的两大难题是歧义消解、未知语言现象。
\begin{itemize}
\item 歧义消解:在自然语言的词法、句法、语义等各个层次中存在大量的歧义现象。比如“什么是一个词”,这就是NLP面临的一个难题。因为不同的人对词语粒度、标准有不同的理解。再比如,语言中存在大量一词多义的现象,在上下文语境中如何准确找到对应的词义,这些都是NLP研究面临的实实在在的难题。
\item 未知语言现象:未知语言现象主要由两个方面的原因导致。第一点,人类语言一直处于不断演化过程中,同一个语言表达,在新的时空环境下,可能已经不再是以前的含义;而且,由于信息网络的发达,人们构造、传播新语言现象的能力大大增强,比如互联网上每天都在涌现新的语言词汇。第二点,在NLP研究中,由于整理收录能力、知识表达能力等现实因素的制约,实际中并没有一种可以准确、全面表达人类语言知识工程或工具。
\end{itemize}

\section{发展历史}
NLP发展历史中存在两种不同的的研究方法:基于规则的理性主义;基于统计的经验主义。它们对语言的不同理解,体现了它们不同的哲学思想。
\begin{itemize}
\item 理性主义:认为自然语言是由语言规则来产生和描述的;因此他们相信,只要能够用适当的形式将人类语言规则表示出来,就能够理解人类语言。
\item 经验主义:认为语言知识可以从语言数据中获取,只要建立有效的统计模型就可以理解语言;因此他们相信,如果有足够多语言数据用于统计,就能够理解人类语言。
\end{itemize}

NLP发展历史可以总结为5个时期:
\begin{enumerate}
\item 经验主义萌芽时期:时间大约到20世纪50年代。这个时期NLP或多或少具有经验主义色彩。例如,1913年马尔科夫提出马尔科夫模型\cite{markov1913example}的时候,就曾经计算过长诗中元音与辅音出现的频度概率;再比如,1948年,香农把离散马尔科夫的概率模型应用于语言自动机\cite{shannon1948mathematical}的时候,也曾统计过英语字母的频率。

\item 经验主义低谷时期:时间从1956年~20世纪90年代。1956年,乔姆斯基首先提出使用有限状态机来刻画自然语言\cite{chomsky1956three}。具体来说,就是使用数学的代数、集合论为基础核心,将各种语言现象统一抽象为代数、集合上的运算规则。形式语言理论影响深远,在此后很长一段时期,很多学者逐步完善并扩展了形式语言理论。这段时期,NLP领域的研究方法几乎完全被理性主义主导,经验主义被打入谷底。

\item 经验主义复苏时期:时间从20世纪50年代末到90年代初期。虽然这期间,例行理性主义占据主流,但是有学者已经开始思考引入基于语料库的统计方法到NLP研究中;这其中的代表是1967年诞生的联机语料库Brown Corpus\cite{kuvcera1967computational}。另外,和NLP紧密相关的机器学习方法得到了较大的发展;比如这段时期,诞生了贝叶斯模型、最大熵\cite{jaynes1957information}、维特比算法\cite{viterbi1967error}、隐马尔可夫模型\cite{stratonovich1965conditional}等等,NLP研究有了更多理论工具可供使用。

\item 经验主义爆发时期:时间20世纪90年代中期~2010年左右。时间处于20世纪90年代前期,此刻经验主义已经处于全面复苏的前夜。一方面,由于机器学习领域诞生了很多新理论和方法,推动了NLP快速发展;另一方面,计算机的存储容量、计算能力已经极大提升,使得很多计算量偏大的机器学习方法逐渐实用。受益于这两方面,经验主义终于开始全面复苏,迎来了一个前所未有的黄金时期。20世纪90年代以后,以语料库和统计学习为基础,基于机器学习的词法分析、句法分析、机器翻译、语音识别等研究不断涌现。Ruder\cite{ruder2018review}在文章中总结了此时期,一些影响力较大里程碑,按照时间线索列举如下:

\begin{itemize}
    \item 1993年,IBM提出了基于语料库的统计机器翻译模型\cite{brown1993mathematics},吹响了经验主义的全面复苏的号角。
    \item 1994年,隐马尔科夫模型在语音识别\cite{lee1989speaker}\cite{deng1994speech}中成功运用,标志了经验主义全面复苏。
    \item 1998年,Berkeley大学主导了FrameNet项目\cite{baker1998berkeley},提出了框架语义学并开放了语料库。这个项目,大大提升了语义角色标注的效果,并在后续促进了,组块分析、依存分析等其他NLP任务的发展。
    \item 2001年,Lafferty发表了条件随机场(Conditional Random Fields,CRF)\cite{lafferty2001conditional}模型。从此,CRF成为序列标注任务的标志性模型,被广泛用于自动分词、词性标注、专名识别等NLP任务中。
    \item 2002年,IBM提出了BLEU\cite{papineni2002bleu},给出了双语互译质量度量的自动评估标准,结束了翻译效果人工评估繁重主观的难题,为后续机器翻译快速发展奠定基础。同年,结构化感知机\cite{collins2002discriminative}的出现,丰富了序列标注方法,很快也被广泛应用序列标注任务中。另外,情感分析\cite{pang2002thumbs}也开始成为NLP研究热点之一。
    \item 2003年,诞生了用于主题建模的标准模型LDA\cite{blei2003latent}(Latent Dirichlet Allocation),它被应用于主题建模、文本分类、相似度计算等方面。
    \item 2004年,Taskar等人提出最大边缘模型\cite{taskar2004max},相比于SVM,它更适合获取结构化数据中的相关性,应用于句法解析任务中。
    \item 2006年,Hovy等人发布大型多语言语料库OntoNotes\cite{hovy2006ontonotes},它已被用于训练、评估多个NLP任务,如专名识别、依存分析等。
    \item 2008年,Milne等人介绍了利用维基百科计算语义相关性\cite{witten2008effective}的方法。维基百科,到目前为止,依旧是NLP语料库最重要成员之一,被广泛用于语言模型、实体链指、知识图谱等各种NLP任务。
    \item 2009年,Go等人提出了远程监督(Distant Supervision)\cite{go2009twitter}的思想,利用维基百科丰富了情感分类标注数据,获得任务效果的较大提升。此后,远程监督思想被更多学者,扩充到关系抽取、信息抽取、情感分类、文本分类等NLP任务中。它们大多通过利用现有大型知识库、语料库(比如维基百科),通过启发式方法,从中抽取大量带有噪声的数据,用于弥补标注数据的不足。
\end{itemize}

\item 经验主义现代时期:时间从2001年左右至今。这段时期,经验主义再上高峰,具有更加鲜明的特色,神经网络、深度学习的是目前NLP研究的关键词。Ruder在博文\cite{ruder2018review}在中总结了这段时期里NLP的研究趋势,归结为8个方面,列举如下:

\begin{itemize}
    \item 神经语言模型(Nerual Language Model,NLM):语言模型是NLP中基础工具,它处理的是给定上文,建模下一个词出现的概率。2003年,Bengio等人提出基于神经网络的语言模型\cite{bengio2003neural},用于替代传统基于计数和平滑的ngram语言模型。此后,NLM被众多研究者进行了扩展,出现了基于标准循环神经网络(RNN)、\cite{mikolov2010recurrent}、长短时记忆网络(LSTM)\cite{graves2013generating}等变种。
    \item 多任务学习(Multi-task Learning):多任务学习通过共享模型参数,使得多个任务可以协同学习、共同提升。2008年,Collobert等首次将多任务学习\cite{collobert2008unified}应用于NLP。到现在,多任务学习在NLP任务中广泛体现,并且很多学者还在不断发展和扩充此思想。
    \item 词嵌入(Word Embedding):在NLP和信息检索领域,人们为了表示一个句子的含义,常常忽略句子的文法和语序,仅考虑频次,采用One-Hot编码,简化成词袋(Bag Of Words)模型进行表达。One-Hot编码有两个缺陷:1. 编码向量高维稀疏;2.不方便计算词之间的相似度。2013年,Mikolov等人提出了词嵌入模型word2vec\cite{mikolov2013efficient},改用低维稠密向量对词语进行编码,获取的词向量蕴含了语言中的近义、同位、上下位等性质。由于词向量蕴含的语义相似度特性,它已成为NLP在深度学习时代的标配,在后续的研究中发挥了重要作用。
    \item NLP神经网络(Neural Networks For NLP):2013年,神经网络开始应用于NLP任务。2013年,Graves等将循环神经网络\cite{graves2013hybrid}应用于语音识别;2013年,Socher 等使用递归神经网络\cite{socher2013recursive},改进了传统的句法分析;2014年,Kalchbrenner 和Kim等人,将卷积神经网络\cite{kalchbrenner2014convolutional}\cite{kim2014convolutional}应用于句子建模和文本分类。至此,神经网络3大结构,都成功应用于NLP中。
    \item Seq2Seq(Sequence To Sequence Model)模型:它是将一个序列映射到另一个序列的通用框架,包含编码器和解码器两部分。编码器将输入句子,压缩为一个固定大小的向量;解码器根据前一步输出和当前状态预测当前位置的输出。2014年,Sutskever等人首次提出了Seq2Seq\cite{sutskever2014sequence}模型框架,将其成功应用于机器翻译任务。它标记着机器翻译,从此进入神经网络机器翻译(Nerual Machine Translation,NMT)时代。后续,很多研究扩展了Seq2Seq的应用范围,产生了比如图像标题生成,文本生成等有趣的应用。
    \item 注意力(Attention)机制:2014年,它由Bahdanau\cite{bahdanau2014neural}等人首次提出。注意力机制,是NMT又一大突破,是NMT能够超过统计机器翻译的关键;它和Seq2Seq一起,将NMT推广到大规模工业应用。通过注意力机制,解码器可以回看编码器的序列状态,极大优化了机器翻译的对齐效果,使得NMT跨进工业应用门槛。后续很多学者,提出了注意力机制的多种实现。注意力机制除了直接激发Transformer\cite{vaswani2017attention}发明,也被广泛应用句法分析\cite{vinyals2015grammar}、阅读理解\cite{hermann2015teaching}等任务。
    \item 基于记忆的网络(Memory-based Networks):由于RNN隐状态、注意力机制等提供的记忆能力不足,无法存储长期、大量的信息。所以,2015年前后,很多学者提出了基于记忆的网络来改进这一问题。其中的代表有,神经图灵机\cite{graves2014neural}、记忆网络\cite{weston2014memory}、端到端记忆网络\cite{sukhbaatar2015end}、动态记忆网络\cite{kumar2016ask}、神经微分计算机\cite{graves2016hybrid})和循环实体网络\cite{henaff2016tracking}。记忆网络技术,广泛适用于机器机器阅读理解、基于知识库的问答、多轮对话系统等方面。
    \item 预训练语言模型(Pretrained Language Models):2018年是预训练语言模型元年,因为从此以后NLP众多任务的研究范式都被改变。2018年,诞生了ELMO\cite{peters2018deep}、GPT\cite{radford2018improving}、BERT\cite{devlin2018bert}这3个杰出代表。它们通过将语言模型嵌入作为特征,使用目标任务数对语言模型对进行微调,使得众多NLP任务都获得了大幅度提升。由于预训练语言模型,只需要无标记的数据便可以进行学习,因此对于标记数据稀缺的低资源场景,预训练语言模型尤其有用。
\end{itemize}

\end{enumerate}

\section{应用领域}
自然语言处理研究的内容非常广泛,应用范围也非常广泛,如下举例一些常见的应用场景:
\begin{itemize}
\item 机器翻译:利用计算机实现自然语言(英语、汉语等)之间的自动翻译。
\item 自动摘要:利用计算机自动地从原始文档中提取全面准确地反映该文档中心内容的简单连贯的短文。
\item 文本分类:在预定义分类体系下,根据文本的特征,将给定文本于一个或多个类别相关联的过程。
\item 情感分类:根据文本所表达的含义和情感将文本划分为褒扬或者贬义的两种或几种类型,是对作者倾向性、观点、态度的划分,因此也称倾向性分析。
\item 信息抽取:从非结构化或半结构的自然语言文本中提取出于某个主题相关的实体、关系。事件等事实信息,并且形成结构化信息输出。
\item 信息检索:用户输入一个表述需求信息的查询字段,系统回复一个包含所需要信息的文档列表。其核心技术在于索引构建和相关性计算。
\item 问答系统:接受用户自然语言形式描述的问题,从大量异构数据中查找或者推断用户问题答案的信息检索系统。
\end{itemize}

